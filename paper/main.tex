\documentclass[runningheads]{llncs}
\usepackage[T1]{fontenc}
\usepackage{graphicx}
\begin{document}

\title{Predicción de crimen con Machine Learning usando datos espaciotemporales
de la región del Callao}

\titlerunning{Predicción de riesgo con ML}
\author{Juan Carlos Molero Rojas}

\authorrunning{J. Molero}
\institute{Universidad Tecnológica del Perú}

\maketitle

\begin{abstract}
\end{abstract}

\section{Introducción}

La seguridad ciudadana se ha transformado en uno de los retos sociales más
importantes en América Latina, que afectan de forma directa la calidad de vida
y el bienestar común. Este concepto incluye no solamente la protección contra
el delito, sino también la seguridad de derechos, la coexistencia pacífica y la
fe en las instituciones. Hoy en día, el aumento de delitos como la violencia
entre personas, el hurto y el crimen organizado han producido una sensación
generalizada de inseguridad, sobre todo en áreas urbanas en situación de
vulnerabilidad.

En el año 2020, un estudio realizado en el marco del proyecto CITYCoP encuestó
a 272 ciudadanos de 11 países europeos, revelando que el 16.2\% de los
participantes percibían vivir en zonas de alta criminalidad, mientras que el
68.8\% afirmaban lo contrario y un 15.1\% no estaban seguros. Aquellos que se
sentían expuestos a altos niveles de criminalidad reportaron un 58\% más de
signos de desorden social y físico en sus barrios, como edificios abandonados,
basura en las calles y actos de vandalismo. Además, presentaron un 37\% menos
de confianza en la policía y un 25\% menos de cohesión comunitaria en
comparación con quienes vivían en zonas consideradas seguras. Estos datos
evidencian que la percepción de inseguridad no solo afecta el bienestar
psicológico de los ciudadanos, sino que también deteriora la confianza
institucional y el tejido social \cite{reid2020}.

En Colombia, la seguridad ciudadana enfrenta una crisis persistente que se
refleja en cifras alarmantes. Solo en el año 2022 se registraron 13,536
homicidios intencionales, lo que equivale a un promedio de 37 muertes violentas
por día. Esta situación afecta a jóvenes y adultos entre los 19 y 59 años,
quienes representan más del 89\% de las víctimas. Asimismo, el hurto a personas
aumentó un 25.4\% respecto al año 2021, alcanzando más de 351,000 casos,
consolidándose como uno de los delitos de mayor crecimiento en la última
década. Con estas cifras se descubren patrones estructurales vinculados a la
informalidad laboral, la pobreza y la limitada presencia institucional en zonas
rurales \cite{nunez2024}.

En el ámbito nacional, la inseguridad ciudadana se ha consolidado como uno de
los principales desafíos sociales, con una percepción negativa que afecta a
múltiples regiones del Perú. Según una revisión sistemática de literatura de 15
estudios realizados entre 2020 y 2024, más del 73\% de los ciudadanos en zonas
como Chiclayo consideran ineficaces las intervenciones policiales, mientras que
en distritos como San Martín de Porres se reporta una falta de efectivos y
demoras considerables en la atención de denuncias. Esta situación se agrava por
factores estructurales como el desempleo juvenil, la informalidad económica, la
corrupción institucional y la migración venezolana, que han intensificado el
temor y la desconfianza hacia las autoridades. La inseguridad no solo deteriora
la calidad de vida, sino que también genera efectos psicológicos y sociales que
afectan la salud mental \cite{anton-chunga2025}.

\section{Trabajos relacionados}

La investigación sobre patrones delictivos es un aspecto importante para la
seguridad pública y, debido al avance de la inteligencia artificial, los
modelos de machine learning son una herramienta excelente para poder investigar
y contener el delito. El objetivo de este estudio es predecir los casos de
delincuencia de 2017 a 2020 utilizando un conjunto de datos de 2001 a 2016,
mediante la creación de modelos predictivos que anticipen futuras tendencias
delictivas. La metodología se basa en un modelo de machine learning que utiliza
algoritmos tanto de clasificación, agrupación y regresión, como lo son los
algoritmos de Regresión Lineal y Random Forest, aplicados a datos de delitos de
la India entre 2001 y 2016. Los resultados muestran una alta precisión en la
predicción de diversos tipos de delitos, con tasas de exactitud que van desde
el 89\% en intentos de asesinato hasta el 95\% en casos de violación y robo. En
conclusión, los modelos de machine learning, en particular el de K-Nearest
Neighbors, demuestran ser eficaces para predecir delitos con la precisión
deseada, identificando y localizando las zonas de mayor incidencia
\cite{s2024}.

Asimismo, dada la alta percepción de inseguridad en Perú, y en particular en el
distrito limeño de Los Olivos que se ha convertido en un problema nacional,
surge esta investigación. El objetivo del estudio fue proponer una solución
tecnológica para la seguridad ciudadana y la prevención del delito, capaz de
enviar una alerta en tiempo real a los usuarios indicando la probabilidad de
que ocurra un acto delictivo en su ubicación. La metodología se basó en el
desarrollo de un modelo de machine learning utilizando el algoritmo Naive
Bayes, el cual fue validado comparando su rendimiento con otros algoritmos como
Classification Forest, Catboost y KNN a través de una matriz de confusión y
métricas como accuracy, precision y recall, además de ser probado con una
muestra de 108 usuarios. Los resultados de la validación con usuarios arrojaron
un alto nivel de aceptación, con un 93.5\% de ellos indicando que la solución
ayuda a prevenir incidentes delictivos. En la validación técnica, aunque el
artículo no especifica los porcentajes exactos de las métricas, según menciona
el autor, se determinó que Naive Bayes fue el más adecuado porque a pesar de
tener un valor de accuracy similar al de otros modelos, ofrecía la mejor
distribución de variables por categoría y una mayor velocidad de predicción. En
conclusión, el proyecto logró desarrollar una solución tecnológica funcional
para la prevención del delito, demostrando la eficacia del algoritmo Naive
Bayes y obteniendo una alta aceptación por parte de los usuarios
\cite{mansilla2023}.

En \cite{ahmad2024} los autores mencionan la importancia de entender los
patrones de los registros delictivos para poder prevenir tipos de crímenes que
vienen desde hurto y vandalismo hasta homicidios y terrorismo. El objetivo de
su investigación fue crear un dashboard basado en datos espacio temporales para
predecir puntos críticos en las ciudades de tal manera que sea posible realizar
un seguimiento en tiempo real. Los autores llamaron CHART a la metodología
utilizada durante su investigación, este consiste en el uso de los datasets
"Crime in Vancouver" de Kaggle con 530 652 registros desde 2006 al 2021, "ICT
Police Crime Data" de la base de datos policial ICT con 120 442 registros desde
2009 al 2021 y "Crime in India" con 530,652 registros, seguido del
preprocesamiento de los datos através de la normalización y asegurando
consistencia en las 3 fuentes de datos; continuando con el balance de los datos
con Adaptive Synthetic Sampling (ADASYN); seguido de el proceso de ingeniería
de características para extraer características importantes, etiquetarlas y
clasificarlas; como últimos pasos, los autores optaron por hacer un mapa de
calor para tener una visualización de los registros delictivos y la aplicación
de el modelo Random Forest para fines predictivos y preventivos. Como
resultado, la metodología utilizada en el estudio obtuvo un 95.65\% de
accuracy, 93.87\% de precisión y un 94.56\% de recall, logrando hacer una
diferencia significativa con las metodologías tradicionales que usan modelos
como KNN, Linear Regression, Logistic Regression, Support Vector Machine y
Naive Bayes. En conclusión, este estudio permite poder reforzar la seguridad
enfocándose en zonas con alto riesgo gracias a la visualización del mapa de
calor y prevenir incidentes mediante el modelo de machine predictivo.

\section{Metodología}

\begin{figure}[h!] 
    \centering 
    \includegraphics[width=1\textwidth]{diagrams/metodologia.png}
    \caption{Metodología}
\end{figure}


\subsection{Obtención del dataset}

Para este estudio, se utilizó el conjunto de datos de acceso público "Registros
delictivos del Observatorio Regional de Seguridad Ciudadana en la Provincia
Constitucional del Callao". Este dataset contiene un total de 2643 registros de
incidentes reportados.

Cada registro detalla información relevante sobre la denuncia, incluyendo la
ubicación geográfica, la marca temporal y la naturaleza del incidente. Las
columnas clave utilizadas para este análisis se describen en la tabla
\ref{tab:campos-dataset-completos}.

\begin{table}[h!]
\centering
\caption{Campos del dataset "Registros delictivos del Observatorio Regional de Seguridad Ciudadana en la Provincia Constitucional del Callao"}
\label{tab:campos-dataset-completos}
\resizebox{\textwidth}{!}{%
\begin{tabular}{l p{10cm}}
\toprule
\textbf{Campo} & \textbf{Descripción} \\
\midrule
\texttt{FECHA\_CORTE} & Fecha de corte de la extracción de datos. \\
\texttt{FECHA\_REGISTRO} & Fecha en que se registró la denuncia. \\
\texttt{ID\_DOC\_DENUNCIA} & Identificador único del documento de denuncia. \\
\texttt{UBIGEO} & Código de Ubicación Geográfica (Ubigeo). \\
\texttt{DEPARTAMENTO} & Departamento donde ocurrió el hecho. \\
\texttt{PROVINCIA} & Provincia donde ocurrió el hecho. \\
\texttt{DISTRITO} & Distrito donde ocurrió el hecho. \\
\texttt{TIPO\_DE\_DENUNCIA} & Tipo de documento (ej. "DENUNCIA"). \\
\texttt{SITUACION\_DENUNCIA} & Estado actual de la denuncia (ej. "PENDIENTE"). \\
\texttt{TIPO} & Categoría principal del incidente (ej. "PATRIMONIO"). \\
\texttt{SUBTIPO} & Subcategoría del incidente. \\
\texttt{MODALIDAD} & Modalidad específica del incidente. \\
\texttt{FECHA\_HECHO} & Fecha en la que ocurrió el incidente. \\
\texttt{HORA\_HECHO} & Hora (en formato entero) del incidente. \\
\texttt{UBICACION} & Dirección o lugar de referencia del incidente. \\
\texttt{DESCRIPCION} & Descripción de la ubicación. \\
\texttt{FECHA\_NACIMIENTO} & Fecha de nacimiento del denunciante/víctima. \\
\texttt{EDAD\_PERSONA} & Edad del denunciante/víctima. \\
\texttt{SEXO} & Sexo del denunciante/víctima. \\
\texttt{ESTADO\_CIVIL} & Estado civil del denunciante/víctima. \\
\texttt{GRADO\_INSTRUCCION} & Grado de instrucción del denunciante/víctima. \\
\texttt{OCUPACION} & Ocupación del denunciante/víctima. \\
\texttt{PAIS\_NATAL} & País de nacimiento del denunciante/víctima. \\
\texttt{MES} & Mes en que ocurrió el hecho. \\
\texttt{LONGITUD} & Coordenada de longitud (geográfica). \\
\texttt{LATITUD} & Coordenada de latitud (geográfica). \\
\bottomrule
\end{tabular}
}
\end{table}

\begin{figure}[h!] 
    \centering 
    \includegraphics[width=1\textwidth]{diagrams/incidentes.png}
    \caption{Número de incidentes por hora}
\end{figure}

\subsection{Preprocesamiento}

El dataset crudo no es directamente aplicable a un modelo de predicción de
riesgo. Por lo tanto, se implementó un pipeline de preprocesamiento para
transformar los registros de incidentes en un formato de (Zona, Hora)
$\rightarrow$ Riesgo.

\subsubsection{Clustering espacial (DBSCAN)}

Para definir "zonas" o "hotspots" de riesgo de manera automática, se aplicó el
algoritmo DBSCAN (Density-Based Spatial Clustering of Applications with Noise).
Este método se aplicó sobre las coordenadas (LONGITUD, LATITUD) escaladas y
permitió agrupar los incidentes en clusters de alta densidad, asignando a cada
uno un CLUSTER\_ID. Los incidentes aislados (ruido, cluster -1) fueron
descartados del análisis posterior.

\begin{figure}[h!] 
    \centering 
    \includegraphics[width=0.8\textwidth]{diagrams/cluster_dbscan.png}
    \caption{Mapa de clusters de riesgo generado por DBSCAN}
\end{figure}

\subsubsection{Generación del Dataset de Entrenamiento}

Se generó un nuevo dataset donde la unidad de análisis no es el incidente, sino
el bloque (Zona, Hora). Se crearon todas las combinaciones posibles de
CLUSTER\_ID (zonas) y HORA\_DEL\_DIA (0-23). Posteriormente, se contabilizó el
número de incidentes (NRO\_INCIDENTES) para cada bloque.

\subsubsection{Codificación de Variables}

la variable objetivo NIVEL\_RIESGO se creó mediante la discretización de
NRO\_INCIDENTES en tres categorías: 'Bajo' (0 incidentes), 'Medio' y 'Alto'
(definidos por la mediana de los bloques con incidentes). Las variables
predictoras (CLUSTER\_ID y HORA\_DEL\_DIA) fueron transformadas usando One-Hot
Encoding para su uso en los modelos.

\subsection{Implementación de los modelos}

Para la tarea de clasificación multiclase (Bajo, Medio, Alto), se seleccionaron
y evaluaron dos modelos con diferentes enfoques algorítmicos. Los datos se
dividieron en conjuntos de entrenamiento (80\%) y prueba (20\%) utilizando
estratificación para mantener la proporción de las clases.

\subsubsection{Random Forest}

Se implementó un Random Forest Classifier, un modelo de ensamblaje (ensemble)
basado en árboles de decisión. Este método opera construyendo múltiples árboles
de decisión durante el entrenamiento y emitiendo la clase que es la moda de las
clases de los árboles individuales. Se utilizó el parámetro
class\_weight='balanced' para mitigar el desbalanceo de clases inherente al
dataset.

\subsubsection{K-Nearest Neighbors}

Se implementó un clasificador K-Nearest Neighbors (KNN). Este es un modelo no
paramétrico basado en instancia que clasifica un punto de datos (un bloque
Zona-Hora) según la clase mayoritaria de sus 'k' vecinos más cercanos en el
espacio de características codificado. Para este estudio, se utilizó un valor
de k=7.

\bibliographystyle{splncs04}
\bibliography{ref}

\end{document}
